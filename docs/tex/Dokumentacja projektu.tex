\documentclass[a4paper, 12pt, oneside]{article}
\usepackage{polski}
\usepackage[utf8]{inputenc}
\usepackage{geometry}
\usepackage{listings}
\usepackage{fancyhdr}
\usepackage{lastpage}
\newgeometry{tmargin=3.5cm, bmargin=3.5cm, lmargin=2.5cm, rmargin=2.5cm}
\linespread{1.5}

\title{\textbf{Timesheet}\linebreak Projekt zaliczeniowy -- dokumentacja}
\author{J.W., O.Z., M.B, R.B, M.S}
\date{2017}

\usepackage{fancyhdr}
 
\pagestyle{fancy}
\fancyhf{}
\rhead{Timesheet}
\lhead{\bfseries\leftmark}
\rfoot{str. \thepage /\pageref{LastPage}}

\begin{document}

% Strona tytułowa

\maketitle
\thispagestyle{empty}

\newpage

% Spis treści

\tableofcontents
\newpage

% Treść

\section{Założenia projektowe}
	\subsection{Przeznaczenie aplikacji}
		\paragraph{}Celem niniejszego projektu jest stworzenie aplikacji webowej, pozwalającej na zarządzanie zadaniami podczas pracy w zespole. Umożliwiać ona będzie tworzenie wielu projektów, a w ich ramach zadań, do realizacji których przydzielać będzie można konkretnych użytkowników. Każdy projekt posiadać będzie menadżera, który ma większe uprawnienia, niż pozostali członkowie zespołu. Administrator aplikacji posiada maksymalne uprawnienia i może zarządzać również menadżerami.
	\subsection{Elementy składowe}
		\paragraph{}Zgodnie z przedstawionymi wymaganiami, tworzona aplikacja składać się będzie z następujących elementów (cech):
		\begin{itemize}
		
		\item Ogólne:
		
			\begin{itemize}
				\item Witryna internetowa,
				\item Należy zastosować ORM lub Micro-ORM,
				\item Dane powinny być walidowane po stronie front i back endu,
				\item Mechanizm logowania błędów,
				\item Projektu musi się znajdować na repozytorium,
				\item Dane muszą być zapisywane i odczytywane z bazy danych,
				\item Projekt będzie oceniany również ze względu na estetykę,
				\item Projekt musi posiadać czytelną dokumentację,
			\end{itemize}
			
		\item Aplikacja:
		
		\begin{itemize}
			\item Rejestracje nowego użytkownika wraz z podaniem danych osobowych,
			\item Mechanizm "Przypominania hasła",
			\item W systemie muszą występować 3 role - Administrator, Manager i Pracownik,
			\item Moduł przesyłania wiadomości pomiędzy użytkownikami systemu,
			\item We wszystkich tabelach w systemie powinna być możliwość filtracji danych,
			\item Należy zaimplementować mechanizm aprobowania wpisów.
		\end{itemize}
		
		\item Pracownik:
			\begin{itemize}
			
			\item Możliwość dodawania i edycji wpisów za danych dzień pracy. Wpis musi zawierać: datę, czas pracy, projekt, zadanie, komentarz,
			\item Wyświetlanie listy wszystkich wpisów (wraz z możliwością ich filtracji).
		
			\end{itemize}
			
			\item Manager:
			
			\begin{itemize}
			
			\item Wyświetlanie, dodawanie, edycja i usuwanie zadań (zadania powinny posiadać przewidywaną estymację),
\item Wyświetlanie wszystkich projektów, które są prowadzone przez danego managera,
\item Wyświetlanie listy pracowników w danym projekcie,
\item Generowanie raportu z przebiegu prac w projekcie,
\item Aprobowanie wpisów pracowników.
			
			\end{itemize}
			
			\item Administrator:
			
			\begin{itemize}
			
			\item Wyświetlanie listy użytkowników, zadań i projektów,
\item Możliwość zablokowania użytkownika lub wymuszenie zmiany hasła,
\item Dodawanie, edycja i usuwanie projektów,
\item Przypisywanie managerów i pracowników do projektów.
			
			\end{itemize}
			
			
			
		\end{itemize}
		
	\section{Specyfikacja techniczna}
		\subsection{Język programowania i framework}
			Pehapiec i laravawawel
		\subsection{Baza danych}
			Jakaż to baza i jej struktura
		\subsection{Repozytorium kodu}
			\paragraph{}
		\subsection{Uruchomienie aplikacji}
			\paragraph{Baza danych}
				Baza chodzi sobie tu i tam (adres i w ogóle)
			\paragraph{Witryna}
				XAMMP-u XAMMP-u
\section{Realizacja}
	\subsection{Logowanie użytkownika}
	\subsection{Rejestracja użytkownika}

\end{document}